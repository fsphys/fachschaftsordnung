\documentclass[a4paper, parskip=half, numbers=noenddot]{scrartcl}

\usepackage{a4wide}
\usepackage{geometry}
\geometry{left=25mm, right=25mm, top=23mm, bottom=25mm}

\usepackage[ngerman]{babel}
\usepackage[T1]{fontenc}

\usepackage{fontspec} % provides font selecting commands
\usepackage{xunicode} % provides unicode character macros
\usepackage{xltxtra}  % provides some fixes/extras

\usepackage{microtype}


\usepackage{hyperref}



\renewcommand{\numberline}[1]{\makebox[2.5em][l]{#1}}


%
% Metadaten
%

\title{Fachschaftsordnung der Fachschaft Physik\\(am Karlsruher Institut für Technologie)}
\author{Version 1.0}
\date{Stand: 26. September 2016}
\hypersetup{
    pdftitle={Fachschaftsordnung der Fachschaft Physik (am Karlsruher Institut für Technologie)},
    pdfauthor={Fachschaft Physik am Karlsruher Institut für Technologie},
    pdfborder={0 0 0.5},
    linkbordercolor={0 0.61 0.50}
}


\begin{document}


%
% Titelseite
%

\maketitle
\thispagestyle{empty}
\vspace{15cm}
%
% Präambel
%
{\bfseries Präambel}\\
Die Fachschaft Physik ist Teil der Verfassten Studierendenschaft des Karlsruher Instituts für Technologie (KIT). Diese Fachschaftsordnung richtet sich nach der Organisationssatzung der Verfassten Studierendenschaft des KIT (Organisationssatzung) und dem Landeshochschulgesetz (LHG). Aus Gründen der besseren Lesbarkeit wird ausschließlich die männliche Form verwendet. Dabei ist jede andere Form impliziert. Die Geschlechtsdefinition obliegt jeder Person selbst.

\pagestyle{empty}
\newpage
% Für Printversion eine leere Seite
\rule{0mm}{0mm}
\newpage



\setcounter{page}{1}
\pagestyle{plain}


%
% Inhaltsverzeichnis
%

%\tableofcontents
%\newpage



%
% Fachschaft
%

\section*{§ 1 Die Fachschaft}
Alle an der KIT-Fakultät für Physik eingeschriebenen Studierenden bilden nach § 65 a (Organisation der Studierendenschaft; Beiträge) Abs. 4 LHG und § 28 (Gliederung, Mitgliedschaft) Organisationssatzung die Fachschaft Physik.


%
% Aufgaben
%

\section*{§ 2 Aufgaben}

Die Fachschaft und ihre Organe haben folgende Aufgaben überparteilich wahrzunehmen:

\begin{enumerate}
    \item die Vertretung der studentischen Interessen insbesondere gegenüber dem KIT und der KIT-Fakultät für Physik,
    
    \item die Förderung aller Studienangelegenheiten,
    
    \item die Studienberatung der Studierenden an der KIT-Fakultät für Physik,
    
    \item die soziale Beratung der Studierenden,
    
    \item die Mitarbeit in den Gremien der KIT-Fakultät für Physik, insbesondere die Mitarbeit an der Gestaltung der Studiengänge, Studien- und Prüfungsordnungen und der Studienbedingungen,
    
    \item die Wahrnehmung der hochschulpolitischen, fachlichen und fachübergreifenden sowie der sozialen, wirtschaftlichen und kulturellen Belange der Studierenden,
    
    \item die Mitwirkung an den Aufgaben des KIT nach §§ 2 bis 7 LHG i. V. m. § 20 Abs. 2 KIT-Gesetz,
    
    \item die Förderung der Gleichstellung und den Abbau von Benachteiligungen innerhalb der Studierendenschaft,
    
    \item die Förderung der sportlichen und musischen Aktivitäten der Studierenden,
    
    \item die Pflege und der Ausbau der überregionalen und internationalen Studierendenbeziehungen,
    
    \item die Information ihrer Mitglieder und der Wissenstransfer innerhalb der Fachschaft,
    \item die Ausrichtung kultureller Veranstaltungen,
    
    \item die Vernetzung innerhalb der Studierendenschaft, insbesondere Teilnahme an der Fachschaftenkonferenz (FSK),
    
    \item die Förderung der politischen Bildung und des staatsbürgerlichen Verantwortungsbewusstseins der Studierenden,
    
    \item die Förderung der Integration ausländischer Studierender, die einen Studienabschluss in Baden-Württemberg anstreben.

\end{enumerate}


%
% Organe
%

\section*{§ 3 Organe der Fachschaft}

Die Organe der Fachschaft sind
\begin{enumerate}
    \item die Fachschaftsversammlung,
    \item die Fachschaftssitzung,
    \item der Fachschaftsvorstand.
\end{enumerate}



%
% Fachschaftsversammlung
%

\section*{§ 4 Fachschaftsversammlung}

(1) Die Fachschaftsversammlung ist das beschließende Organ der Fachschaft.\\

(2) Jedes Fachschaftsmitglied ist auf der Fachschaftsversammlung stimm- und antragsberechtigt.\\

(3) Die Fachschaftsversammlung wird mindestens einmal pro Semester sowie auf Antrag von mindestens 5 \% der Fachschaftsmitglieder vom Fachschaftsvorstand einberufen. Bei der Einberufung muss eine Tagesordnung vorgeschlagen werden.\\

(4) Die Fachschaftsversammlung ist beschlussfähig, wenn ordnungsgemäß einberufen wurde und mindestens sechs Fachschaftsmitglieder anwesend sind.\\

(5) Der Fachschaftsvorstand schlägt eine Sitzungsleitung vor, die von der Versammlung bestätigt wird.\\

(6) Die Fachschaftsversammlung soll an einem Vorlesungstag stattfinden.\\

(7) Die Fachschaftsversammlung muss mindestens sieben Tage im Voraus per Aushang einberufen werden.\\

(8) Aufgaben:
\begin{enumerate}
\item Beschluss und Änderung der Fachschaftsordnung,
\item Genehmigung des Haushaltsplans der Fachschaft,
\item Einsetzen des Wahlleiters,
\item Bestätigung der Vertreter in der FSK,
\item Beschluss einer Neuwahl des Fachschaftsvorstands gemäß Absatz 12,
\item Erstellung des Wahlvorschlags zum Fachschaftsvorstand.
\end{enumerate}

(9) Alle weiteren Aufgaben und Beschlüsse können von der Fachschaftssitzung übernommen werden. Beschlüsse der Fachschaftsversammlung heben widersprechende Beschlüsse der Fachschaftssitzung auf.\\

(10) Die Fachschaftsversammlung beschließt mit relativer Mehrheit gemäß § 41 (Mehrheiten) Organisationssatzung.\\

(11) Änderungen und Beschlüsse zur Fachschaftsordnung müssen mit Zweidrittelmehrheit der abgegebenen Stimmen gemäß § 41 (Mehrheiten) Organisationssatzung beschlossen werden.\\

(12) Die Fachschaftsversammlung kann mit 10 \% aller Stimmen und Zweidrittelmehrheit der abgegebenen Stimmen gemäß § 41 (Mehrheiten) Organisationssatzung beschließen, eine Neuwahl des Fachschaftsvorstands zu veranlassen.\\

(13) Das Protokoll muss innerhalb von zwei Wochen unter Berücksichtigung des Datenschutzes öffentlich und von der Sitzungsleitung unterschrieben ausgehängt werden.\\


%
% Fachschaftssitzung
%

\section*{§ 5 Fachschaftssitzung}

(1) Die Fachschaftssitzung entscheidet über alle Angelegenheiten der Fachschaft im Rahmen der von der Fachschaftsversammlung beschlossenen Vorgaben.\\

(2) Sie soll in der Vorlesungszeit wöchentlich stattfinden.\\

(3) Die Fachschaftssitzung wird mit einer Frist von drei Werktagen vom Fachschaftsvorstand per Aushang einberufen.\\

(4) Die Fachschaftssitzung ist öffentlich. Alle Anwesenden haben Rederecht.\\

(5) Alle Fachschaftsmitglieder haben Stimm- und Antragsrecht.\\

(6) Die Fachschaftssitzung ist beschlussfähig, wenn mindestens sechs Mitglieder anwesend sind.\\

(7) Zu Beginn wird eine Sitzungsleitung bestimmt. Im Falle von Uneinigkeit wird diese mit relativer Mehrheit gemäß § 41 (Mehrheiten) Organisationssatzung gewählt.\\

(8) Die Fachschaftssitzung beschließt mit relativer Mehrheit gemäß § 41 (Mehrheiten) Organisationssatzung.\\

(9) Das Protokoll wird von der Fachschaftssitzung auf der nächsten Sitzung bestätigt und soll bis zu deren Ende unter Berücksichtigung des Datenschutzes veröffentlicht werden.\\\\

(10) Die Fachschaftssitzung empfiehlt dem Vorstand Vertreter für die Fachschaftenkonferenz sowie das beratende Mitglied im KIT-Fakultätsrat.



%
% Fachschaftsvorstand
%

\section*{§ 6 Fachschaftsvorstand}

(1) Der Fachschaftsvorstand ist das ausführende Organ der Fachschaft.\\

(2) Der Fachschaftsvorstand besteht aus den Fachschaftssprechern und den studentischen KIT Fakultätsratsmitgliedern.\\

(3) Aufgaben

\begin{enumerate}
    \item  Einberufung der Fachschaftsversammlungen und Fachschaftssitzungen,
    \item Wahl eines Fachschaftsmitgliedes, das mit beratender Stimme an den Sitzungen des KIT-Fakultätsrates teilnimmt,
    \item Verantwortung für die Umsetzung von Beschlüssen tragen,
    \item als Ansprechpartner die Beschlüsse und Meinungen der Fachschaft kommunizieren.
\end{enumerate}

(4) Der Vorstand entsendet mindestens einen Vertreter in die FSK. Diese müssen von der
Fachschaftsversammlung bestätigt werden.\\
(5) Der Vorstand beschließt und wählt mit absoluter Mehrheit gemäß § 41 (Mehrheiten) Or-
ganisationssatzung.\\
(6) Der Vorstand ist der Fachschaftsversammlung und der Fachschaftssitzung rechenschaftspflichtig.


%
% Fachschaftssprecher
%

\section*{§ 7 Fachschaftssprecher}

(1) Die Fachschaftssprecher sind die fünf Vertreter mit den meisten Stimmen bei der Wahl nach § 30 (Fachschaftsvorstand) Abs. 2 Organisationssatzung.\\

(2) Ein Fachschaftssprecher scheidet aus dem Amt
\begin{enumerate}
    \item am Ende der Amtsperiode,
    \item durch Exmatrikulation,
    \item bei Wahl eines neuen Vorstandes nach § 4 Abs. 12.
\end{enumerate}
(3) Bei Ausscheiden eines Fachschaftssprechers rückt der Kandidat mit den nächstmeisten
Stimmen nach. Steht kein Kandidat mehr zur Verfügung, bleibt das Amt unbesetzt. Fällt die
Anzahl der Fachschaftssprecher unter zwei, ist eine Fachschaftsversammlung von dem noch
verbleibenden Fachschaftssprecher innerhalb von zwei Wochen in der Vorlesungszeit einzuberufen, um über Neuwahlen zu entscheiden. Ist der Fachschaftsvorstand unbesetzt, so ist vom Ältestenrat eine Fachschaftsversammlung einzuberufen, um eine Neuwahl einzuleiten.


%
% Finanzen
%

\section*{§ 8 Finanzen}

(1) Die Fachschaftssitzung wählt einen Finanzreferenten entsprechend § 16 (Fachschaftsfinanzen) Abs. 3 der Finanzordnung der Verfassten Studierendenschaft des KIT.\\

(2) Die Amtszeit des Finanzreferents beträgt ein Jahr und beginnt mit der Wahl.\\

(3) Ein Finanzreferent scheidet aus dem Amt
\begin{enumerate}
    \item am Ende der Amtszeit,
    \item durch Exmatrikulation,
    \item durch eigenen Verzicht,
    \item durch Neuwahl.
\end{enumerate}

(4) Der Finanzreferent regelt die Finanzen der Fachschaft und muss den Finanzausschuss der Verfassten Studierendenschaft unterstützen.\\

(5) Der Finanzreferent stellt den Haushaltsplan der Fachschaft auf, der von der Fachschaftsversammlung genehmigt werden muss.\\

(6) Der Finanzreferent ist der Fachschaftssitzung und der Fachschaftsversammlung rechenschaftspflichtig.\\

(7) Weiteres regelt die Finanzordnung der Verfassten Studierendenschaft des KIT.\\

(8) Mittel der Fachschaft dürfen nur für die Zwecke gemäß dieser Ordnung verwendet werden.\\

(9) Die Ausgaben der Fachschaft müssen durch die Fachschaftssitzung beschlossen werden.


%
% Inkrafttreten
%


\section*{§ 9 Inkrafttreten}

Diese Fachschaftsordnung tritt am Tag nach ihrer Bekanntmachung in den Amtlichen Bekanntmachungen des KIT in Kraft.\\\\
Karlsruhe, den 26. September 2016\\
Prof. Dr.-Ing. Holger Hanselka\\
(Präsident)


\end{document}